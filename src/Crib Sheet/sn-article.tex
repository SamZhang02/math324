\documentclass[sn-mathphys,Numbered]{sn-jnl}% Math and Physical Sciences Reference Style

\usepackage{graphicx}
\usepackage{multirow}
\usepackage{amsmath,amssymb,amsfonts}
\usepackage{amsthm}
\usepackage{mathrsfs}
\usepackage[title]{appendix}
\usepackage{xcolor}
\usepackage{textcomp}
\usepackage{manyfoot}
\usepackage{booktabs}
\usepackage{algorithm}
\usepackage{algorithmicx}
\usepackage{algpseudocode}
\usepackage{listings}

\newcommand{\Var}{\mathbb{V}}
\newcommand{\E}{\mathbb{E}}
\newcommand{\Bias}{\mathbb{B}}
\newcommand{\Remark}{\textbf{Remark:}}
\newcommand{\Example}{\textbf{Example:}}
\newcommand{\Proof}{\textbf{Proof:}}
\newcommand{\QED}{\hfill $\blacksquare$}

%% as per the requirement new theorem styles can be included as shown below
% \theoremstyle{thmstyleone}%
% \newtheorem{theorem}{Theorem}%  meant for continuous numbers
%%\newtheorem{theorem}{Theorem}[section]% meant for sectionwise numbers
%% optional argument [theorem] produces theorem numbering sequence instead of independent numbers for Proposition
% \newtheorem{proposition}[theorem]{Proposition}% 
%%\newtheorem{proposition}{Proposition}% to get separate numbers for theorem and proposition etc.

% \theoremstyle{thmstyletwo}%
% \newtheorem{example}{Example}%
% \newtheorem{remark}{Remark}%

% \theoremstyle{thmstylethree}%
% \newtheorem{definition}{Definition}%

% \raggedbottom
%%\unnumbered% uncomment this for unnumbered level heads

\begin{document}

\title[MATH324]{MATH324 Crib Sheet}

\author{\fnm{Sam} \sur{Zhang}}

\maketitle
\newpage

\section{Properties of Estimators and Statistics}
\subsection{Biasedness}\label{biasedness}
An estimator $\hat{\theta}$ is biased if $\E(\hat{\theta}) \neq \theta$.\\
Example:\\ $\hat{\theta} = \frac{1}{n} \sum_{i=1}^n X_i$ is biased if $X_i$ are
i.i.d. with mean $\mu$ and variance $\sigma^2$.
\begin{align}
    \E ( \hat{\theta} ) = \frac{1}{n} \sum_{i=1}^n \E (X_i
    ) = \frac{1}{n} n \mu = \mu
\end{align}
The bias is determined by the equation
\begin{equation}
    \Bias ( \hat{\theta} ) = \E ( \hat{\theta} ) - \theta
\end{equation}
We can generally find an unbiased estimator from a biased estimator by eliminating
the constants surrounding the estimator, such that $\E(\hat{\theta}) \to \theta$.

\subsection{Consistency}\label{consistency}
An estimator $\hat{\theta}_n$ is consistent if $\hat{\theta}$ converges in
probability to $\theta$ as $n \rightarrow \infty$.
\begin{align}
    \lim_{n \rightarrow \infty} P(|\hat{\theta}_n - \theta| > \epsilon) = 0 \\
    \lim_{n \rightarrow \infty} P(|\hat{\theta}_n - \theta| \leq \epsilon) = 1
\end{align}
This is equivalent to the following:
\begin{align}
    \lim_{n \to \infty} \Var ( \hat{\theta_n} ) = 0
\end{align}
\subsection{Asymptotic Normality}\label{normality}
An estimator $\hat{\theta}$ is asymptotically normal if $\hat{\theta}$
converges in distribution to a normal distribution as $n \rightarrow \infty$.\\
Example:\\ $\bar X = \frac{1}{n} \sum_{i=1}^n X_i$ is asymptotically normal if
$X_i$ are independent and identically distributed (i.i.d.) with mean $\mu$ and
variance $\sigma^2$.
\begin{align}
    \bar{X}                                         & \sim N(\mu, \frac{\sigma^2}{n}) \\
    \frac{\bar{X} - \mu}{\sqrt{\frac{\sigma^2}{n}}} & \sim N(0, 1)
\end{align}

\subsection{Efficiency}\label{efficiency}
An estimator $\hat{\theta}$ is efficient if $\hat{\theta}$ has the smallest
variance among all unbiased estimators of $\theta$.
\subsubsection{The Rao Blackwell Theorem}\label{rao-blackwell}
Let $\hat{\theta}$ be an unbiased estimator of $\theta$ such that
$\Var(\hat{\theta}) < \infty$. If $U$ is a sufficient statistic for $\theta$,
define $\hat \theta ^ * = \E (\hat \theta | U)$. Then $\forall \theta$:
\begin{equation}
    \E(\hat \theta ^ *) = \theta \quad \text{and} \quad \Var(\hat \theta ^ *) \leq
    \Var(\hat \theta)\nonumber
\end{equation}
\textbf{Remark}: The result of the Rao Blackwell Theorem is the
\textit{minimum-variance unbiased estimator} of $\theta$. (MVUE)

\subsection{Sufficiency}\label{sufficiency}
A statistic $T$ is sufficient for $\theta$ if $T$ contains all the information
about $\theta$. This implies that $\theta$ can be uniquely determined from an
estimator based on $T$ without any loss of information.\\ Example:\\ $T = \bar
    X$ is sufficient for $\mu$ if $X_i$ are i.i.d. with mean $\mu$ and variance
$\sigma^2$.

\section{Hypothesis Testing}\label{hypothesis-testing}
\subsection{The T-test}\label{t-test}
\subsubsection{Large-Sample Hypothesis Testing}\label{large-sample-testing}
\subsubsection{Small-Sample Hypothesis Lemma}\label{small-sample-testing}

\subsection{The Likelihood Ratio Test}\label{likelihood-ratio}
\subsubsection{The Likelihood Ratio Test for a Single Parameter}\label{single-param}
\textbf{The Neyman-Pearson Lemma}
\subsubsection{The Likelihood Ratio Test for Multiple Parameters}\label{multiple-param}

\section{Linear Regression}\label{linear-regression}
\subsection{Parameters of a Linear Model}\label{linear-model}
\subsection{The Least Squares Estimator}\label{least-squares}
\subsection{The correlation coefficient}\label{correlation}
\subsection{Hypothesis Testing for Linear Regression}\label{linear-hypothesis}
\subsubsection{The T-test}\label{regression-t-test}
\subsubsection{The F-test}\label{regression-f-test}

\end{document}